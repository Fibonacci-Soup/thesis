%!TEX root = ../thesis.tex
%*******************************************************************************
%*********************************** Conclusion Chapter *****************************
%*******************************************************************************

\chapter{Conclusion and Outlook}

The research presented in this thesis has focused on advancing the field of computer-generated holography (CGH) through the development and optimisation of phase-only holograms. By exploring and implementing various algorithms, we have identified the strengths and weaknesses of each approach in terms of computational efficiency and reconstruction quality.

Chapter 3 proposed the Digital Pre-Distorted One-Step Phase Retrieval (DPD-OSPR) algorithm, described the experimental setup, explained the method for determining the DPD curve, and then applied the DPD curve to improve holographic projection quality. The results demonstrated that DPD can significantly enhance the quality of the reconstructed images. On the grey-scale ramp target, the DPD-OSPR method reduced the NMSE by a stunning 95.45\%. Then the DPD-OSPR was applied on two sample images, it was observed that more details were shown in the replay field, and the NMSE's of the two example images were reduced by 19.86\% and 15.64\% respectively. As the DPD is a one-to-one mapping, the extra computation required is negligible. The effectiveness of the proposed DPD-OSPR method to improve reconstruction quality on the existing OSPR algorithm while still keeping its ability for real-time holography was demonstrated.

Chapter 4 focused on the optimisation of phase-only holograms, and proved the effectiveness of using the Limited-memory Broyden-Fletcher-Goldfarb-Shanno (L-BFGS) algorithm for CGH. Then the novel Target Image Phase Optimisation (TIPO) technique was proposed, which optimises the phase of the target image instead of the phase of the hologram. The L-BFGS algorithm was shown to offer efficient convergence, improving the quality of the generated holograms. Then the two existing 3D CGH optimisation methods, the sum-of-hologram (SoH) and sum-of-loss(SoL) techniques were investigated. The novel method using L-BFGS optimiser with Sequential Slicing (SS) technique was proposed to generate phase-only hologram for multi-depth target, which is faster than SoL and of better quality than SoH. The L-BFGS with SS technique has demonstrated a good suppression on the quality imbalance across the multi-depth slices, benefiting from the nature of L-BFGS being a second order optimiser, which implicitly records the historical gradients by other slices for the determination of the descent direction. For both GD and L-BFGS optimisation algorithms, the SS technique runs faster and produces better reconstruction quality than the simple SoH technique, and it is much quicker than the SoL technique especially when the number of slices get large. The proposed SS method also proved effective for complicated 3D targets and demonstrated great ability of time-limited applications.

Chapter 5 proposed the Multi-Frame Holograms Batched Optimisation (MFHBO) algorithm to generate multi-frame binary-phase holograms to be displayed on the high-refresh-rate binary-phase SLM in the lab. By comparing simulation and optical experiment results, the proposed MFHBO method was shown to have superior performance in reducing noise and improving reconstruction quality for multi-frame holograms than the existing multi-frame binary-phase holograms generation methods OSPR and AD-OSPR on the holographic projector with binary-phase SLM, for all the single-slice far-field targets and the multi-slice near-field targets tested. Although the proposed MFHBO method is slower than the existing OSPR and AD-OSPR methods, its much better reconstruction quality makes it especially suitable for pre-computed high-quality hologram applications. Its strong advantage for high contrast target also makes it well-suited for photo-lithography applications. The proposed method can also be adapted for multi-level SLM's in the future once high-refresh-rate high-resolution SLM's are available.

Finally, Chapter 6 explored the information capacity of phase-only CGH. This chapter examined quantisation effects on hologram bit depth and their impact on reconstruction quality, and reached the conclusion that, holograms with higher bit depth can sustain more information therefore producing better quality reconstructions. However, the quality of the reconstruction is not correlated to either the entropy or the delentropy of the target image, so neither entropy nor delentropy can quantify how difficult an image is for phase-only hologram generation. Additionally, the entropy of the hologram generated using quantised GS algorithm is not only bounded by the hologram bit depth, but also affected by the entropy of the initial phase. For applications where holograms file size is a high priority, it is advised to start with a low entropy initial phase rather than a random initial phase and it is recommended to reduce the hologram bit depth limit. In future work, a suitable metric will be the goal to quantify how difficult any image is for phase hologram generation.

Overall, this thesis has contributed a series of novel phase retrieval techniques, with improvements achieved in the computational speed and reconstruction quality of CGH. While current methods provide a foundation for practical implementations, there still remains substantial room for improvement, particularly in achieving real-time, high-quality holography. Future research should continue to build on these findings, exploring novel algorithms and advancing technologies to realise the full potential of CGH.
