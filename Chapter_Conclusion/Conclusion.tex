%!TEX root = ../thesis.tex
%*******************************************************************************
%*********************************** Conclusion Chapter *****************************
%*******************************************************************************

\chapter{Summary and Future Work}
\section{Summary}
The research presented in this thesis has focused on advancing the field of computer-generated holography (CGH) through the development and optimisation of phase-only holograms. By exploring and implementing various algorithms, this thesis has identified the strengths and weaknesses of each approach in terms of computational efficiency and reconstruction quality.

Chapter 3 proposed the Digital Pre-Distorted One-Step Phase Retrieval (DPD-OSPR) algorithm, described the experimental setup, explained the method for determining the DPD curve and then applied the DPD curve to improve holographic projection quality. The results demonstrated that DPD can significantly enhance the quality of the reconstructed images. On the grey-scale ramp target, the DPD-OSPR method reduced the NMSE by a stunning 95.45\%. Then the DPD-OSPR was applied on two sample images. It was observed that more details were shown in the replay field, and the NMSE's of the two example images were reduced by 19.86\% and 15.64\% respectively. As the DPD is a one-to-one mapping, the extra computation required is negligible. The effectiveness of the proposed DPD-OSPR method to improve reconstruction quality on the existing OSPR algorithm while still keeping its ability for real-time holography was demonstrated.

Chapter 4 focused on the optimisation of phase-only holograms, and proved the effectiveness of using the Limited-memory Broyden-Fletcher-Goldfarb-Shanno (L-BFGS) algorithm for CGH. Then the novel Target Image Phase Optimisation (TIPO) technique was proposed, which optimises the phase of the target image instead of the phase of the hologram. The L-BFGS algorithm was shown to be able to offer efficient convergence, improving the quality of the generated holograms. Then the two existing 3D CGH optimisation methods - the sum-of-hologram (SoH) and sum-of-loss(SoL) techniques - were investigated. The novel method using L-BFGS optimiser with Sequential Slicing (SS) technique was proposed to generate phase-only hologram for multi-depth target, which was shown to be faster than the SoL technique and have better quality than the SoH technique. The L-BFGS with SS technique has demonstrated a good suppression on the quality imbalance across the multi-depth slices, benefiting from the nature of L-BFGS being a second order optimiser, which implicitly records the historical gradients by other slices for the determination of the descent direction. For both GD and L-BFGS optimisation algorithms, the SS technique runs faster and produces better reconstruction quality than the simple SoH technique. The proposed SS method also proved effective for complicated 3D targets and demonstrated great ability of time-limited applications.

Chapter 5 proposed the Multi-Frame Holograms Batched Optimisation (MFHBO) algorithm to generate multi-frame binary-phase holograms to be displayed on the high-refresh-rate binary-phase SLM in the lab. Comparisons of simulation and optical experiment results showed that the proposed MFHBO method had superior performance in reducing noise and improving reconstruction quality for multi-frame holograms than did the existing multi-frame binary-phase holograms generation methods OSPR and AD-OSPR on the holographic projector with binary-phase SLM, for all the single-slice far-field targets and the multi-slice near-field targets tested. Although the proposed MFHBO method is slower than the existing OSPR and AD-OSPR methods, its much better reconstruction quality makes it especially suitable for pre-computed high-quality hologram applications. Its strong advantage for high contrast target also makes it well-suited for photo-lithography applications. The proposed method can also be adapted for multi-level SLM's in the future once high-refresh-rate multi-level phase SLM's are available.

Finally, Chapter 6 explored the information capacity of phase-only CGH. This chapter examined quantisation effects on hologram bit depth and their impact on reconstruction quality, and reached the conclusion that holograms with higher bit depth can sustain more information and can therefore produce reconstructions with better quality. However, the quality of the reconstruction is not correlated to either the entropy or the delentropy of the target image, so neither entropy nor delentropy can quantify how difficult an image is for phase-only hologram generation. Moreover, the entropy of the hologram generated using quantised GS algorithm is not only bounded by the hologram bit depth, but also affected by the entropy of the initial phase. For applications where holograms file size is a high priority, it is advised to start with a low entropy initial phase rather than a random initial phase, and it is recommended to reduce the hologram bit depth limit.

In conclusion, this thesis has contributed a series of novel phase retrieval techniques, with improvements achieved in the computational speed and reconstruction quality of phase-only CGH. While current methods provide a foundation for practical implementations, there still remains substantial room for improvement, particularly in achieving real-time, high-quality holography. Future research should continue to build on these findings and explore novel algorithms and advancing technologies to realise the full potential of CGH.

\section{Future Work}
The experimental work in this thesis in \cref{chapter:Digital Pre-Distorted One-Step Phase Retrieval (DPD-OSPR) Algorithm} and \cref{chapter:Multi Frame Holograms Batched Optimisation} are based on a high-refresh-rate binary-phase SLM (as introduced in \cref{sec:experimental setup}). In the future, if high-refresh-rate multi-level-phase modulators become available, it will be recommended to adapt the MFHBO method proposed in \cref{chapter:Multi Frame Holograms Batched Optimisation} to generate multi-frame multi-level phase holograms by simply setting the quantisation step in \cref{fig:MFHO_flowchart} to match the bit-depth level of the SLM. With a higher bit-depth level of the SLM, the reconstruction quality can be further improved. Currently available SLMs on the market not only lack in bit-depth levels and refresh rates, but also lack in the size and the number of pixels. If an SLM of the size of a regular computer monitor (e.g. more than 13 inch) having trillions of pixels were to be available in the future, it should have similar or better visual effects compared to the analogue film-based holograms such as the one in \cref{fig:Dennis-Gabor-Hologram-2}, but with computer-generated contents.

In \cref{chapter:Multi Frame Holograms Batched Optimisation}, the MFHBO method was proposed to generate multi-frame binary-phase holograms for high-refresh-rate binary-phase SLMs. It provided good reconstruction quality but the computational speed was much slower than the existing binary phase hologram generation methods. In the future, it would be ideal to further improve the computational speed of the MFHBO method, so that it can be used for real-time applications. The proposed MFHBO method also consumes a lot of memory, as it requires storing and optimising the holograms for each frame in the batch, which is not suitable for applications with limited memory resources. Future work could focus on reducing the memory footprint of the MFHBO method, for example by optimising the holograms in a pipeline manner instead of storing all frames in memory at once.

In \cref{chapter:information capacity}, attempts were made to find a metric to quantify how difficult any image is for phase hologram generation, with entropy and delentropy being ruled out. In the future, more research can be done to find a suitable metric to predict phase hologram generation difficulty. If found, it will set a general standard on how CGH algorithms perform and how far away they are from the fundamental limit. The analogy is information entropy in data compression, where entropy denotes the lower bound of compression rate for any compression algorithm (i.e. a metric is needed to indicate the limit of reconstruction quality of the phase hologram generated for any target image). However, if a fully complex modulator could be built in the future, all the issues with phase retrieval algorithms can be avoided. Before the invention of a fully complex SLM, the current phase-only SLMs still has rooms of improvements as they are currently pixelated and arranged in a grid, leading to errors and noise in quantisation and discretisation. The gaps between the adjacent pixels reduce the active area of the SLM and cause mis-matches between simulations and optical experiments. In the future, it would be ideal to develop a new type of non-pixelated SLM that can modulate the phase continuously. It could be addressed acoustically, for example a thin layer of liquid that is modulated by ultrasonic waves. The continuous phase modulation will eliminate the quantisation and discretisation errors, and the continuous phase profile will also eliminate the gaps between pixels, leading to 100\% active area of the SLM and achieve better matches between simulations and optical experiments.