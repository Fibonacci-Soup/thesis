% ************************** Thesis Abstract *****************************
% Use `abstract' as an option in the document class to print only the titlepage and the abstract.
\begin{abstract}
Currently popular three-dimensional (3D) imaging technologies on the consumer market, including 3D cinema, 3D TV, handheld 3D devices (e.g. Nintendo 3DS, HTC Evo 3D) and Virtual Reality (VR) and Augmented Reality (AR) head sets are stereoscopic displays, which fail to provide a natural experience. These systems rely on artificial tricks to simulate depth, leading to discomfort and limited realism. Holography, by contrast, fully reconstructs the optical wavefront of 3D objects and generates a full 3D light field with true depth cues, which has the potential to provide an unparalleled sense of depth and immersion. However, the practical realisation of holographic displays faces significant challenges due to the limitations of existing Computer-Generated Holography (CGH) methods, which often involve trade-offs between computational efficiency and reconstruction quality. This thesis tackles these challenges through developing novel algorithmic approaches aimed at overcoming the obstacles that prevent the holography technology from dominating the 3D display market. The generation of holograms for practical displays is constrained by the need to compute phase-only holograms compatible with current spatial light modulators (SLMs), which are not yet capable of displaying fully complex holograms but can only modulate either phase or amplitude, among which phase modulation is usually preferred. A method of computing phase-only CGH is called phase retrieval. This thesis proposes several novel phase retrieval methods for holographic displays. The Digital Pre-Distorted One-Step Phase Retrieval (DPD-OSPR) method is proposed to mitigate non-linearities in the existing holographic projection system. The proposal of the Limited-memory Broyden-Fletcher-Goldfarb-Shanno (L-BFGS) optimisation algorithm together with the Sequential Slicing (SS) technique offers an efficient approach for 3D phase-only CGH by evaluating individual depth slices iteratively as opposed to evaluating the entire volume. The Multi-Frame Holograms Batched Optimisation (MFHBO) algorithm is proposed to improve the perceived image quality. Additionally, the information capacity of phase-only CGH is explored from an information-theory perspective to investigate the fundamental limits of phase-only CGH.
\end{abstract}
